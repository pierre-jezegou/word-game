\documentclass[10pt,a4paper,hidelinks]{report}
\usepackage[utf8]{inputenc}
\usepackage[french]{babel}
\usepackage[T1]{fontenc}

\newcommand{\documentStatus}{DRAFT}


\usepackage{amsmath}
\usepackage{amsfonts}
\usepackage{amssymb}
\usepackage{graphicx}
\usepackage{lmodern}
\usepackage{tikz}
\usetikzlibrary{positioning}
\usepackage{epigraph} 
\usepackage[left=2.5cm,
            right=2.5cm,
            top=2cm,
            bottom=2cm]{geometry}
\usepackage{setspace}
\usepackage{caption}
\usepackage{subcaption}
\usepackage{epigraph}
\usepackage{pdflscape}

\usepackage{titlesec}
\usepackage{tcolorbox}
\usepackage{background}
\usepackage{url}
\usepackage[pdfauthor={Pierre Jézégou},
            pdftitle={ADS assignement},
            pdfsubject={Word games},
            pdfkeywords={}]{hyperref}
\usepackage{wrapfig}


\backgroundsetup{contents=\documentStatus, color=\watermarkColor}

\usepackage{fancyhdr}
\usepackage{textpos}
\usepackage{sectsty}
\usepackage{xcolor}

\setlength{\parindent}{0pt}

%%%%%%%%%%%%%%% Colors %%%%%%%%%%%%%%%
\subsectionfont{\color{fib_red}}
\subsubsectionfont{\color{fib_gray}}
\renewcommand\fbox{\fcolorbox{black}{fib_red!20}}

\definecolor{fib_red}{RGB}{191,21,64}
\definecolor{fib_gray}{RGB}{111,111,111}


\usepackage[Bjornstrup]{fncychap}
\newcommand{\watermarkColor}{red!10}


\onehalfspacing


\newcommand\VRule[1][\arrayrulewidth]{\vrule width #1}
\usepackage{xcolor,colortbl}

\newtcolorbox{mybox}[1]{
    arc=0pt,
    boxrule=0pt,
    colback=#1,
    width=\linewidth,
    halign=left,
}

\newenvironment{framed}[2]{
    \vspace*{0.5em}
    \begin{mybox}{orange!10}
        \textbf{#1} :\hfill \textit{#2}\\
        \hrule\vspace*{1em}
}
{\end{mybox}}

\newenvironment{exercise_description}[1]{
    \begin{framed}{Exercice description}{#1}
}
{\end{framed}}

\usepackage{lmodern}
\renewcommand*\familydefault{\sfdefault}


\begin{document}
\pagestyle{plain}
\backgroundsetup{contents=,color=red!30}
\pagecolor{white}

\begin{center}
    \color{red!50!white}
    \textbf{\huge{STATUT - \documentStatus}}
\end{center}

\vfill

\color{black}
\begin{center}
    % \includegraphics[width=0.5\linewidth]{images/logos/fib.png} \\
    \includegraphics[height=2cm]{images/logos/upc_logo.jpeg} \\
    \vfill

    \rule{\linewidth}{0.5mm} \\[1cm]
    {\Huge \textsc{\textcolor{fib_red}{Word Games}}}\\[1cm]
    {\Large \textsc{Assignment}}\\[0.4cm]
    {\huge \textsc{\textbf{Advanced data structures}}}\\[1cm]
    {\Large \textsc{Master in Research and Innovation - UPC}}\\[0.4cm]
    \rule{\linewidth}{0.5mm} \\[1.5cm]
\end{center}

\vfill

\textbf{Authors :}
\begin{itemize}
\item Pierre \textsc{Jézégou}\newline
\textit{(Engineering student at École Centrale de Lille, Exchange student at UPC)}
\end{itemize}

\vfill

\newpage
\color{black}
\pagecolor{white}
\pagestyle{fancy}

\section{Introduction}


This program implements a trie data structure to store words.
A trie is a tree-like data structure that stores a dynamic set of strings, where each node represents a single character of the string.
The TrieNode class represents the structure of a trie node, which contains information such as its children, whether it marks the end of a word, its value (character), and its depth in the trie.\\

The Trie class implements the operations and building procedure for the trie. It has methods to insert words into the trie and to search for words in the trie.
The program also includes functions to generate TikZ code for visualizing the trie. The \verb|generate_tikz_trie| function generates TikZ code to visualize the trie structure, while the \verb|generate_search_path| function generates TikZ code to visualize the search path for a given word in the trie.
Finally, the program creates an instance of the Trie class, inserts words from a given dictionary into the trie, and generates TikZ code to visualize the trie with a specific search word highlighted.

\begin{exercise_description}{Word challenge}
    In Word Challenge the user gives a (multi)set of up to 17 letters, and the program produces all words which can be written using a subset of the given letters. For example, if the user gives the letters \verb|{E,T,F,H,R,R,E,O,E}| the program will write by increasing length and then in alphabetic order all the words that can be built using (some) of these letters, for example, \verb|FOR, HER, ORE, THE, .., HERE, ..., THEREFORE|. The dictionary does only contain words of $\text{length}\geqslant  3$, hence if given $\ell \geqslant 3$ the list of results starts with words of length 3 and ends with words of length $\ell$ (or smaller, if there were no words of length $\ell$ using all given letters). Besides being able to play interactively with the user, the program must give an "automatic mode" in which the program does the following repeatedly:
    \begin{enumerate}
    \item Picks a random word from the dictionary of given length $\ell$
    \item Rearranges its letters randomly
    \item Supplies these letters to the function that generates the list of words which can be built from the given letters
    \end{enumerate}
    In automatic mode the user gives the number of times that the game will be played, and the length $\ell$, and it outputs the average number of words found and the average CPU time that the program takes to "solve" a word of length $\ell$.
\end{exercise_description}

\begin{figure}[h]
    \centering
    \begin{tikzpicture}[scale=0.8]
\node [circle, fill=green!70!black!70]{}[sibling distance=11cm]
	child{node[circle, fill=black!30]{F}[sibling distance=6cm]
		child{node[circle, fill=black!30]{O}[sibling distance=1cm]
			child{node[circle, fill=black!30]{R}[sibling distance=1cm]
				child{node[circle, fill=black!30]{C}[sibling distance=1cm]
					child{node[circle, fill=black!30]{E}[sibling distance=1cm]
					}
				}
				child{node[circle, fill=black!30]{W}[sibling distance=1cm]
					child{node[circle, fill=black!30]{A}[sibling distance=1cm]
						child{node[circle, fill=black!30]{R}[sibling distance=1cm]
							child{node[circle, fill=black!30]{D}[sibling distance=1cm]
								child{node[circle, fill=black!30]{E}[sibling distance=1cm]
									child{node[circle, fill=black!30]{R}[sibling distance=1cm]
									}
								}
							}
						}
					}
				}
			}
		}
		child{node[circle, fill=black!30]{I}[sibling distance=1cm]
			child{node[circle, fill=black!30]{R}[sibling distance=1cm]
				child{node[circle, fill=black!30]{E}[sibling distance=1cm]
					child{node[circle, fill=black!30]{W}[sibling distance=1cm]
						child{node[circle, fill=black!30]{O}[sibling distance=1cm]
							child{node[circle, fill=black!30]{R}[sibling distance=1cm]
								child{node[circle, fill=black!30]{K}[sibling distance=1cm]
								}
							}
						}
					}
				}
				child{node[circle, fill=black!30]{M}[sibling distance=1cm]
				}
				child{node[circle, fill=black!30]{S}[sibling distance=1cm]
					child{node[circle, fill=black!30]{T}[sibling distance=1cm]
						child{node[circle, fill=black!30]{L}[sibling distance=1cm]
							child{node[circle, fill=black!30]{Y}[sibling distance=1cm]
							}
						}
						child{node[circle, fill=black!30]{S}[sibling distance=1cm]
						}
					}
				}
			}
		}
	}
	child{node[circle, fill=green!70!black!70]{H}[sibling distance=1cm]
		child{node[circle, fill=green!70!black!70]{E}[sibling distance=1cm]
			child{node[circle, fill=black!30]{R}[sibling distance=1cm]
				child{node[circle, fill=black!30]{E}[sibling distance=1cm]
				}
			}
			child{node[circle, fill=black!30]{Y}[sibling distance=1cm]
			}
			child{node[circle, fill=black!30]{A}[sibling distance=1cm]
				child{node[circle, fill=black!30]{T}[sibling distance=1cm]
				}
			}
			child{node[circle, fill=green!70!black!70]{I}[sibling distance=1cm]
				child{node[circle, fill=green!70!black!70]{G}[sibling distance=1cm]
					child{node[circle, fill=green!70!black!70]{H}[sibling distance=1cm]
						child{node[circle, fill=green!70!black!70]{T}[sibling distance=1cm]
							child{node[circle, fill=black!30]{Y}[sibling distance=1cm]
							}
							child{node[circle, fill=black!30]{E}[sibling distance=1cm]
								child{node[circle, fill=black!30]{E}[sibling distance=1cm]
									child{node[circle, fill=black!30]{N}[sibling distance=1cm]
									}
								}
							}
						}
					}
				}
			}
		}
	}
;
\end{tikzpicture}


    \caption{Search path}
    \label{fig:search_path_pgm}
\end{figure}

\begin{figure}[h]
    \centering
    \begin{tikzpicture}[scale=0.8]
\node [circle, fill=blue!50]{}[sibling distance=11cm]
	child{node[circle, fill=red!50.000000]{F}[sibling distance=6cm]
		child{node[circle, fill=red!25.000000]{O}[sibling distance=1cm]
			child{node[circle, fill=red!16.666667]{R}[sibling distance=1cm]
				child{node[circle, fill=red!12.500000]{C}[sibling distance=1cm]
					child{node[circle, fill=red!10.000000]{E}[sibling distance=1cm]
					}
				}
				child{node[circle, fill=red!12.500000]{W}[sibling distance=1cm]
					child{node[circle, fill=red!10.000000]{A}[sibling distance=1cm]
						child{node[circle, fill=red!8.333333]{R}[sibling distance=1cm]
							child{node[circle, fill=red!7.142857]{D}[sibling distance=1cm]
								child{node[circle, fill=red!6.250000]{E}[sibling distance=1cm]
									child{node[circle, fill=red!5.555556]{R}[sibling distance=1cm]
									}
								}
							}
						}
					}
				}
			}
		}
		child{node[circle, fill=red!25.000000]{I}[sibling distance=1cm]
			child{node[circle, fill=red!16.666667]{R}[sibling distance=1cm]
				child{node[circle, fill=red!12.500000]{E}[sibling distance=1cm]
					child{node[circle, fill=red!10.000000]{W}[sibling distance=1cm]
						child{node[circle, fill=red!8.333333]{O}[sibling distance=1cm]
							child{node[circle, fill=red!7.142857]{R}[sibling distance=1cm]
								child{node[circle, fill=red!6.250000]{K}[sibling distance=1cm]
								}
							}
						}
					}
				}
				child{node[circle, fill=red!12.500000]{M}[sibling distance=1cm]
				}
				child{node[circle, fill=red!12.500000]{S}[sibling distance=1cm]
					child{node[circle, fill=red!10.000000]{T}[sibling distance=1cm]
						child{node[circle, fill=red!8.333333]{L}[sibling distance=1cm]
							child{node[circle, fill=red!7.142857]{Y}[sibling distance=1cm]
							}
						}
						child{node[circle, fill=red!8.333333]{S}[sibling distance=1cm]
						}
					}
				}
			}
		}
	}
	child{node[circle, fill=red!50.000000]{H}[sibling distance=1cm]
		child{node[circle, fill=red!25.000000]{E}[sibling distance=1cm]
			child{node[circle, fill=red!16.666667]{R}[sibling distance=1cm]
				child{node[circle, fill=red!12.500000]{E}[sibling distance=1cm]
				}
			}
			child{node[circle, fill=red!16.666667]{Y}[sibling distance=1cm]
			}
			child{node[circle, fill=red!16.666667]{A}[sibling distance=1cm]
				child{node[circle, fill=red!12.500000]{T}[sibling distance=1cm]
				}
			}
			child{node[circle, fill=red!16.666667]{I}[sibling distance=1cm]
				child{node[circle, fill=red!12.500000]{G}[sibling distance=1cm]
					child{node[circle, fill=red!10.000000]{H}[sibling distance=1cm]
						child{node[circle, fill=red!8.333333]{T}[sibling distance=1cm]
							child{node[circle, fill=red!7.142857]{Y}[sibling distance=1cm]
							}
							child{node[circle, fill=red!7.142857]{E}[sibling distance=1cm]
								child{node[circle, fill=red!6.250000]{E}[sibling distance=1cm]
									child{node[circle, fill=red!5.555556]{N}[sibling distance=1cm]
									}
								}
							}
						}
					}
				}
			}
		}
	}
;
\end{tikzpicture}


    \caption{Draw trie}
    \label{fig:draw_pgm}
\end{figure}

\end{document}
