\usepackage{amsmath}
\usepackage{amsfonts}
\usepackage{amssymb}
\usepackage{graphicx}
\usepackage{lmodern}
\usepackage{tikz}
\usetikzlibrary{positioning}
\usepackage{epigraph} 
\usepackage[left=2.5cm,
            right=2.5cm,
            top=2cm,
            bottom=2cm]{geometry}
\usepackage{setspace}
\usepackage{caption}
\usepackage{subcaption}
\usepackage{epigraph}
\usepackage{pdflscape}

\usepackage{titlesec}
\usepackage{tcolorbox}
\usepackage{background}
\usepackage{url}
\usepackage[pdfauthor={Pierre Jézégou},
            pdftitle={ADS assignement},
            pdfsubject={Word games},
            pdfkeywords={}]{hyperref}
\usepackage{wrapfig}


\backgroundsetup{contents=\documentStatus, color=\watermarkColor}

\usepackage{fancyhdr}
\usepackage{textpos}
\usepackage{sectsty}
\usepackage{xcolor}

\setlength{\parindent}{0pt}

%%%%%%%%%%%%%%% Colors %%%%%%%%%%%%%%%
\subsectionfont{\color{fib_red}}
\subsubsectionfont{\color{fib_gray}}
\renewcommand\fbox{\fcolorbox{black}{fib_red!20}}

\definecolor{fib_red}{RGB}{191,21,64}
\definecolor{fib_gray}{RGB}{111,111,111}


\usepackage[Bjornstrup]{fncychap}
\newcommand{\watermarkColor}{red!10}


\onehalfspacing


\newcommand\VRule[1][\arrayrulewidth]{\vrule width #1}
\usepackage{xcolor,colortbl}

\newtcolorbox{mybox}[1]{
    arc=0pt,
    boxrule=0pt,
    colback=#1,
    width=\linewidth,
    halign=left,
}

\newenvironment{framed}[2]{
    \vspace*{0.5em}
    \begin{mybox}{orange!10}
        \textbf{#1} :\hfill \textit{#2}\\
        \hrule\vspace*{1em}
}
{\end{mybox}}

\newenvironment{exercise_description}[1]{
    \begin{framed}{Exercice description}{#1}
}
{\end{framed}}